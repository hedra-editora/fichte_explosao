PREFÁCIO

\emph{A história da sensibilidade} de Hubert Fichte é, devido a sua
morte precoce em 1986, um projeto literário inacabado, mas ainda assim
peculiar e completamente novo. O projeto de Fichte abarca uma viagem,
que jamais foi concluída, ao redor do mundo, sobretudo ao longo das
marcações geográficas que desde Paul Gilroy definimos como Black
Atlântico. O referido projeto une uma prosa autobiográfica experimental
praticada ao longo da vida inteira à tentativa de repensar a ciência de
pessoas marcadas pelas impressões do holocausto, da bomba atômica, do
colonialismo e da fome global. Ainda que com lacunas, o ciclo romanesco
abrange 18 volumes, entre eles alguns textos ensaísticos, críticas
literárias, entrevistas políticas e tratados etnográficos, por exemplo
acerca das religiões sincréticas da América do Sul e da psiquiatria na
África Ocidental.

Fichte é um autor de interesses bem variados, que giram sobretudo em
torno de sua própria situação e de suas obsessões -- tendo crescido como
meio-judeu, homem gay na Alemanha pós-fascista, ele ama as culturas
africanas e afrodiaspóricas, e vê ainda assim a impossibilidade de
superar as limitações da colonialidade, das formas de conhecimento
estabelecidas da antropologia e da etnologia e da política supostamente
antirracista. Fichte se interessa pela resistência sem violência, pela
luta política e pelo turismo, sobretudo o turismo sexual -- descreve
suas possibilidades e suas catástrofes. Fichte jamais encontrou o método
claro, a perspectiva confiável que procurou desde o princípio. Em alguns
de seus romances, volumes ensaísticos e peças radiofônicas que fazem
parte do projeto, ele se desespera ante sua própria posição
intermediária entre o escritor altamente subjetivo e poético e o
cientista paciente, o jornalista político e o observador inclemente de
si mesmo: sua percepção, sua sexualidade, sua curiosidade.

Sua linguagem é concisa, objetiva, inexorável, mas também transbordante,
musical, gritantemente cômica e estranha. A luta com e contra as aporias
das formas de conhecimento e comunicação e seu próprio envolvimento, ele
a leva adiante de modo bem fichteano, ora sarcástico, ora amargo,
agressivo e sempre pronto a não esconder os seu próprios limites.

``Hubert Fichte: Amor e Etnologia'' descreve e persegue parte da viagem
literária de Fichte, enquanto tenta tacar as aporias externas e
inclusive as mais facilmente endereçáveis. O fundamento do projeto de
vários anos é a tradução de alguns dos romances de Fichte às línguas dos
lugares dos quais eles tratam: o português, o português brasileiro, o
espanhol chileno, o francês, o uolofe, o inglês americano. Sobre a base
das referidas traduções foi iniciado um intercâmbio com artistas dos
mencionados espaços linguísticos, através de curadoras e curadores
convidados, que redundarão em exposições em Lisboa, Salvador, Rio de
Janeiro, Santiago, Nova York, Dacar e Berlim, que por sua vez
acompanharão a publicação das traduções. O projeto ``Hubert Fichte: Amor
e Etnologia'' está sendo levado a cabo pela Casa das Culturas do Mundo
em colaboração com o Instituto Goethe e com o apoio da Fundação S.
Fischer.

\emph{Explosão} é o volume VII de \emph{A história da sensibilidade.}
Ele apresenta diversas viagens do alter ego de Fichte, o escritor Jäcki,
e sua companheira, a fotógrafa Irma (ou seja, Leonore Mau), sobretudo ao
Brasil, mas também à Argentina, ao Chile e às Ilhas da Páscoa. O espaço
narrado se estende da primeira viagem ao Rio de Janeiro, em 1969, à
descrição de uma longa estada nos anos 1971 e 1972 sobretudo em
Salvador, na Bahia, até a volta ao Brasil nos anos de 1980, sobretudo a
São Luís do Maranhão e outros lugares no Norte do país. Em uma inserção
narrativa, Fichte ainda conta como, já no leito de morte, tenta integrar
ao romance a história de sua viagem ou da viagem de Jäcki ao Chile, que
culmina em uma entrevista com Salvador Allende, feita não apenas pelo
Jäcki ficcional, mas também e de fato pelo Fichte real. Fichte autorizou
que o texto fosse manuscrito em uma anotação de 22 de janeiro de 1986. O
manuscrito foi decifrado e ditado em seguida por Leonore Mau. Mais
detalhes a respeito podem ser encontrados na Nota Editorial de Ronald
Kay, que concluiu a edição do presente volume de \emph{A história da
sensibilidade} em março de 1993, e nas minuciosas explicações de Marcelo
Backes nas notas de rodapé à obra.

Diedrich Diederichsen e Anselm Franke

Para o projeto \emph{Explosão}: Max Jorge Hinderer Cruz e Amilcar Packer
